\documentclass[12pt]{article}
\usepackage[margin=1in]{geometry}
\usepackage{lmodern}
\usepackage{setspace}
\usepackage{enumitem}
\setstretch{1.3}
\usepackage{hyperref}

\title{Algorithmic Game Theory: HW2}
\author{Hossein Shahabadi}
\date{April 2025}

\begin{document}

\maketitle

\section*{Exercise 1}

\textbf{Question:} Suggest at least two ways to modify the Olympic badminton tournament format to reduce or eliminate the incentive for teams to intentionally lose a match.

\textbf{Answer:} During the 2012 Olympic badminton tournament, some teams deliberately lost matches to get easier opponents in later rounds. This led to disqualification of eight players.

Two suggestions to address this issue:
\begin{enumerate}[label=\arabic*.]
  \item \textbf{Use a single-elimination format:} In this structure, a loss means elimination. Thus, teams always have strong incentives to win every match. This format was used before 2012 and avoids strategic losing.
  \item \textbf{Random draw for knockout stage:} If keeping a group stage, perform a random draw after the group phase to determine knockout stage matchups. This way, teams can’t manipulate future opponents through losing.
\end{enumerate}

These changes align incentives and encourage fair competition.

\section*{Exercise 2}

\textbf{Question:} Compare the eBay auction with the second-price (Vickrey) auction. Should you bid differently in these two types of auctions?

\textbf{Answer:} The eBay auction is an example of an indirect mechanism, using proxy bidding where users set a maximum bid and the system automatically increases it incrementally.

Despite the difference in format, both eBay auctions and second-price sealed-bid auctions function similarly in principle: the winner pays the second-highest bid.

Therefore, from a strategic standpoint, bidders should behave the same in both: the weakly dominant strategy is to bid your true valuation. In practice, however, due to misunderstanding of the system, eBay bidders may act differently.

\section*{Exercise 3}

\textbf{Question:} Suppose there are $k$ identical items and $n > k$ bidders. Each bidder can receive at most one item. What is the analog of the second-price auction in this case? Prove that your proposed auction is DSIC.

\textbf{Answer:} This scenario maps to the $k$-unit Vickrey auction (uniform-price auction), which works as follows:

\begin{enumerate}[label=\arabic*.]
  \item Collect all bids and sort them in descending order: $b_1 > b_2 > \ldots > b_n$.
  \item The top $k$ bidders each receive one item.
  \item Each winner pays the $(k+1)$-th highest bid.
\end{enumerate}

\textbf{Why this is DSIC (Dominant Strategy Incentive Compatible):}
\begin{itemize}
  \item If a winner increases their bid, the outcome doesn't change.
  \item If they lower their bid, they might lose the item.
  \item If a losing bidder bids lower, the outcome remains unchanged.
  \item If a losing bidder overbids and wins, they may end up overpaying.
\end{itemize}

Hence, truthful bidding maximizes or preserves utility for all participants. Therefore, this auction mechanism is DSIC.

\section*{Exercise 4}

\textbf{Question:} Suppose you want to hire a contractor to perform some task, like remodeling a house. Each contractor has a private cost for performing the task. Give an analog of the Vickrey auction in which contractors report their costs and the auction chooses a contractor and a payment. Truthful reporting should be a dominant strategy in your auction and, assuming truthful bids, your auction should select the contractor with the smallest private cost.

\textbf{Answer:} This setup describes a \textit{reverse auction}, also known as a \textit{procurement auction}. The analog of the Vickrey auction is as follows:

\begin{enumerate}[label=\arabic*.]
  \item Each contractor submits a sealed bid representing their private cost.
  \item The contractor with the \textbf{lowest bid} is selected.
  \item The selected contractor is paid an amount equal to the \textbf{second-lowest bid}.
\end{enumerate}

\textbf{Why this is DSIC (Dominant Strategy Incentive Compatible):}
\begin{itemize}
  \item If a contractor bids truthfully and wins, they are paid more than or equal to their cost.
  \item If they overbid, they might lose the job to a higher-cost contractor.
  \item If they underbid and win, they risk being paid less than their true cost, resulting in a loss.
\end{itemize}

Thus, truthful bidding is the best strategy for all contractors, and the mechanism selects the lowest-cost contractor. Therefore, this reverse Vickrey auction is DSIC.

\section*{Exercise 5}

\textbf{Question:} Consider the following auction. Assume we have $k$ units of a good and $n > k$ participants. Given a reserve price $R$, the auction proceeds as follows:

After all participants submit their bids $b_1, b_2, \dots, b_n$, the organizer selects the top $k$ bidders (set $S$).

While there exists a bidder $i \in S$ with $b_i < R/|S|$, that bidder is removed from $S$, and the condition is rechecked for the remaining set.

Final allocation: If any bidders remain in $S$, each receives one unit of the good and pays $R/|S|$.

\textbf{(a) Show that this allocation rule is monotone.}

\textbf{Answer:} An allocation rule is monotone if increasing a bidder's bid does not reduce their chances of receiving an item. In this auction:

\begin{itemize}
  \item Increasing a bid cannot cause a bidder to drop out of the top $k$.
  \item If a bidder is already in $S$, increasing their bid increases the likelihood that $b_i \geq R/|S|$.
  \item Therefore, increasing a bid cannot result in exclusion from the final allocation.
\end{itemize}

Hence, the allocation rule is monotone.

\textbf{(b) Consider the auction from Exercise 3. Show that if in that auction, the organizer receives at least $R$ units from the participants, then in this auction, the organizer receives exactly $R$ units.}

\textbf{Answer:} In Exercise 3, the $k$-unit Vickrey auction collects total revenue $k \cdot b_{k+1}$. If this amount is at least $R$, then it means the group of top $k$ bidders are willing to pay collectively at least that much.

In the current auction, we begin by selecting the top $k$ bidders. Each of them would be asked to pay $R/k$. If any of them bid less than this threshold, they are removed and the set $S$ shrinks. As $|S|$ becomes smaller, the threshold $R/|S|$ increases.

This iterative removal continues until we find a subset of bidders who are each willing to pay at least $R/|S|$. The key idea is that the auction adjusts the number of winners downward to match the reserve requirement.

As a result, the organizer will collect exactly $R$, because:
\[ 
|S| \cdot \left(\frac{R}{|S|}\right) = R 
\]

Therefore, if the Vickrey auction generates at least $R$, this auction can select a subset of high bidders to meet that exact revenue target.

\section*{Exercise 6}

\textbf{Question:} Consider a Vickrey auction with multiple bidders. Suppose a subset $S$ of the bidders colludes, meaning they submit coordinated false bids to maximize their collective profit. Find the necessary and sufficient conditions for the set $S$ (in terms of the bidders' private values) under which the bidders in $S$ can increase their collective profit by submitting dishonest bids.

\textbf{Answer:} In a Vickrey auction, truthful bidding is a dominant strategy for individuals, but the mechanism is not group-strategyproof. A coalition $S$ can benefit from coordinated misreporting of their values only under specific conditions.

Let:
\begin{itemize}
  \item $v_i$: true valuation of bidder $i$,
  \item $i^* \in S$: the member of $S$ with the highest valuation,
  \item $v_{\text{max}}^{(-S)} = \max_{j \notin S} v_j$: the highest valuation outside of $S$,
  \item $v_{\text{2nd}}$: the second-highest valuation in the truthful auction.
\end{itemize}

\textbf{Necessary and sufficient conditions for profitable collusion:}
\begin{enumerate}
  \item A member of $S$ would win truthfully:
  \[
  v_{i^*} > v_{\text{max}}^{(-S)}
  \]
  \item The second-highest bid in the truthful auction comes from a member of $S$:
  \[
  v_{\text{max}}^{(-S)} < v_{\text{2nd}}
  \]
\end{enumerate}

\textbf{Explanation:} The first condition guarantees that the winner would remain a member of $S$, even without collusion. The second condition ensures that a colluding bidder is responsible for the price paid in the auction. By lowering this second-highest bid, the coalition can reduce the payment made by the winner and thereby increase their collective utility. If both conditions hold, the coalition can profit through coordinated false bidding.

\section*{Exercise 7}

\textbf{Question:} In Myerson's theorem, show that if an allocation rule is not monotone, then it is not implementable.

\textbf{Answer 1:} Myerson’s Lemma states that in single-parameter domains, an allocation rule is implementable (i.e., part of a truthful mechanism) if and only if it is monotone non-decreasing in each agent’s bid.

To show that non-monotonicity implies non-implementability, assume the contrary: suppose an allocation rule $x_i(b_i)$ is not monotone for some agent $i$. That is, there exist two bids $b_i < b_i'$ such that:
\[
x_i(b_i') < x_i(b_i)
\]

This implies that by increasing their bid, agent $i$ decreases their allocation — which discourages them from bidding truthfully.

In such a case, there exists no payment rule that can make truth-telling a dominant strategy for agent $i$. Thus, the mechanism cannot be incentive compatible.

\textbf{Conclusion:} Non-monotone allocation rules cannot be implemented truthfully, which confirms that monotonicity is a necessary condition for implementability as stated by Myerson’s theorem.



\textbf{Answer 2:} To prove that a non-monotone allocation rule is not implementable, we will use the "payment difference sandwich" method from Myerson’s Lemma.

\textbf{Step 1: Payment Difference Sandwich.}  
Let \( z \) and \( y \) be two bid values where \( z < y \). The payment difference sandwich states that for an arbitrary allocation rule \( x(\cdot) \):
\[
z(x(y) - x(z)) \leq p(y) - p(z) \leq y(x(y) - x(z))
\]
This inequality compares the payments for two different bids, \( z \) and \( y \), and shows that the difference in payments depends on the change in the allocation rule.

\textbf{Step 2: Non-Decreasing Allocation Rule.}  
The above inequality implies:
\[
z(x(y) - x(z)) \leq y(x(y) - x(z)) \Rightarrow (y - z)(x(y) - x(z)) \geq 0
\]
This means that for the allocation rule \( x(\cdot) \) to be implementable, it must be \textbf{non-decreasing}. In other words, if a bidder increases their bid, their allocation should either stay the same or improve, never worsen.

\textbf{Step 3: Non-Monotonicity Leads to Non-Implementability.}  
If the allocation rule is non-monotone, there exists a case where increasing a bidder’s bid results in them losing their allocation, violating the condition that increasing the bid should never result in a worse outcome. This breaks the \textbf{incentive compatibility} condition, as bidders would have no dominant strategy to bid truthfully. Therefore, non-monotone allocation rules cannot be implemented.

Thus, if an allocation rule is not monotone, it is not implementable.

\section*{Exercise 8}

\textbf{Question:}  
We concluded the proof of Myerson’s Lemma by giving a “proof by picture” that coupling a monotone and piecewise constant allocation rule $\mathbf{x}$ with the payment formula
\[
p_i(b_i, \mathbf{b}_{-i}) = \sum_{j=1}^{\ell} z_j \cdot [\text{jump in } x_i(\cdot, \mathbf{b}_{-i}) \text{ at } z_j],
\]
yields a DSIC mechanism. Where does the proof-by-picture break down if the piecewise constant allocation rule $\mathbf{x}$ is not monotone?

\textbf{Answer:} Myerson’s Lemma relies critically on the monotonicity of the allocation rule $\mathbf{x}$. The "proof-by-picture" interpretation assumes that the allocation probability $x_i(b_i, \mathbf{b}_{-i})$ is non-decreasing in $b_i$, ensuring the payment formula
\[
p_i(b_i, \mathbf{b}_{-i}) = \sum_{j=1}^{\ell} z_j \cdot [\text{jump in } x_i(\cdot, \mathbf{b}_{-i}) \text{ at } z_j],
\]
aligns with the DSIC property. If $\mathbf{x}$ is \textbf{not monotone}, the following breakdowns occur:  

\begin{enumerate}
  \item \textbf{Negative Payments from Downward Jumps:}  
  Monotonicity ensures all jumps in $x_i(\cdot)$ are non-negative. Without it, increasing $b_i$ might \textit{decrease} $x_i$, creating negative jumps (e.g., $0.6 \to 0.2$). These negative terms in the payment formula lead to $p_i < 0$, meaning the mechanism \textit{pays the agent}. This violates the foundational assumption that agents pay the mechanism, not vice versa.

  \item \textbf{Incentive Misalignment and DSIC Failure:}  
  DSIC requires truthful bidding to maximize utility. A non-monotone $\mathbf{x}$ allows regions where a \textit{lower bid} grants a \textit{higher allocation}. For example, an agent might reduce their bid to exploit a higher allocation probability while avoiding payment spikes. The payment rule (derived for monotone $\mathbf{x}$) fails to penalize such manipulation, breaking incentive compatibility.

  \item \textbf{Collapse of the “Area Under the Curve” Interpretation:}  
  For monotone $\mathbf{x}$, payments correspond to the area under $x_i(\cdot)$ up to $b_i$ (integrating marginal contributions). Non-monotonicity inverts this relationship: downward jumps subtract from the payment, making the interpretation nonsensical. This misaligns payments with the intended incentive structure.
\end{enumerate}

In essence, monotonicity ensures that:
\begin{itemize}
  \item Payments are non-negative and flow from agents to the mechanism.
  \item Truthful bidding is optimal (DSIC).
  \item The payment rule has a valid geometric/economic interpretation.
\end{itemize}

\pagebreak
\textbf{Example of Violation}  

Consider a single-item auction with one bidder and a non-monotone allocation rule:


\[
x(b) = \left\{\begin{array}{@{}lr@{}}
    0.6, & \text{if\hspace{0.2cm}}b \in \left[0, 5\right)\\
    0.2, & \text{if\hspace{0.2cm}}b \in \left[5, 10\right]\\
    \end{array}
\]


\begin{itemize}
  \item \textbf{Payment at $b = 10$:}  
  \[
  p(10) = 5 \cdot (0.2 - 0.6) = -2 \quad (\text{bidder\,receives\,\$2}).
  \]
  
  \item \textbf{DSIC Violation:} For a bidder with value $v = 6$:
  \begin{itemize}
    \item Truthful bid ($b = 6$): Utility $= (0.2 \cdot 6) - (-2) = 3.2$.
    \item Under-report ($b = 4$): Utility $= (0.6 \cdot 6) - 0 = 3.6$.
  \end{itemize}
  The bidder gains by lying, violating DSIC.
\end{itemize}

\textbf{Conclusion}  

Non-monotonicity disrupts the core assumptions of Myerson’s Lemma, enabling negative payments and strategic manipulation. The example illustrates how downward jumps in $\mathbf{x}$ destroy the alignment between payments, allocations, and incentives.


\end{document}
